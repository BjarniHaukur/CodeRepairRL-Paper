\chapter{Implementation and Experimental Work}
\label{ch:work}

This chapter focuses on the practical implementation challenges and experimental execution process that transformed the theoretical framework presented in Chapter~\ref{ch:method} into a working research platform.
We examine the engineering insights gained during implementation, the experimental methodology employed, and the lessons learned from deploying online \ac{RL} for coding agents at scale.

\section{Implementation Challenges and Engineering Insights}
\label{sec:implementation-challenges}

Transforming the theoretical framework from Chapter~\ref{ch:method} into a production system revealed several critical implementation challenges that significantly influenced our experimental design and results.

\subsection{Agent Integration Complexities}

Integrating the Nano agent (detailed in Section~\ref{sec:nano-agent}) with online \ac{RL} training presented unexpected engineering challenges:

\textbf{Trajectory Serialization Overhead}: Converting complex agent interactions into \ac{RL} training data proved computationally expensive, requiring careful optimization of data structures and serialization protocols to prevent bottlenecks in the training pipeline.

\textbf{State Management Complexity}: Maintaining consistent repository states across concurrent agent sessions while enabling rollbacks and resets demanded sophisticated containerization strategies that significantly impacted system throughput.

\textbf{Tool Output Variability}: The stochastic nature of shell command outputs (timing variations, environment differences) introduced unexpected variance in agent trajectories, requiring careful handling to ensure reproducible training results.

\subsection{Scaling Challenges}

Deploying the system at the scale required for meaningful \ac{RL} training revealed several practical limitations:

\textbf{Container Orchestration Overhead}: Managing hundreds of concurrent Docker containers stressed our compute infrastructure beyond initial capacity estimates, requiring significant optimization of resource allocation and cleanup procedures.

\textbf{Memory Management}: Long agent trajectories with extensive command outputs quickly exceeded \ac{GPU} memory limits, forcing careful implementation of trajectory truncation and context management strategies.

\textbf{Fault Tolerance Requirements}: At scale, container failures, network timeouts, and resource exhaustion became common occurrences, necessitating robust error handling and recovery mechanisms not present in initial prototypes.

\subsection{Training Pipeline Adaptations}

\textbf{\ac{GRPO} Modifications}: The standard \ac{GRPO} implementation required significant adaptation for coding agents (as detailed in Section~\ref{sec:rl-algorithm}), particularly in handling variable-length trajectories and tool-calling sequences that differ substantially from typical text generation tasks.

\textbf{Batch Composition Challenges}: Constructing training batches with sufficient diversity while maintaining computational efficiency proved more complex than anticipated, requiring dynamic batching strategies that balance trajectory length, complexity, and reward distribution.

\textbf{Convergence Monitoring}: Traditional \ac{RL} metrics proved insufficient for monitoring training progress, leading to development of domain-specific metrics tracking agent exploration patterns, tool usage effectiveness, and incremental improvement signals.

\subsection{Infrastructure Integration Insights}

\textbf{vLLM Adaptation}: Integrating the inference server architecture (described in Section~\ref{sec:training-infrastructure}) with live weight updates proved more challenging than anticipated, requiring extensive debugging of \ac{CUDA} memory coordination and process synchronization issues.

\textbf{Performance Bottlenecks}: Initial implementations suffered from significant throughput limitations due to inefficient batching of variable-length agent requests, leading to substantial re-engineering of request handling and memory management strategies.

\textbf{Stability Challenges}: Maintaining system stability during continuous training and inference required careful monitoring and automatic recovery mechanisms, as the combined workload stressed infrastructure components in unexpected ways.

\section{Experimental Execution Methodology}
\label{sec:experimental-execution}

\subsection{Training Orchestration}

Executing online \ac{RL} experiments required careful orchestration of multiple system components operating in parallel:

\textbf{Resource Coordination}: Balancing compute allocation between training and inference workloads while maintaining system responsiveness demanded dynamic resource management strategies that evolved throughout the experimental process.

\textbf{Experiment Monitoring}: Tracking training progress across distributed components required comprehensive logging and monitoring infrastructure, with real-time visualization of key metrics including agent success rates, exploration patterns, and system utilization.

\textbf{Checkpoint Management}: Coordinating model checkpoints across training and inference systems while maintaining experiment continuity posed significant logistical challenges, particularly during long training runs spanning multiple days.

\subsection{Data Collection and Quality Assurance}

\textbf{Trajectory Validation}: Ensuring the quality and consistency of collected agent trajectories required extensive validation procedures, including automated detection of truncated sessions, malformed tool calls, and inconsistent state transitions.

\textbf{Reward Computation Validation}: Verifying the correctness of reward calculations (detailed in Section~\ref{sec:reward-design}) across thousands of agent trajectories demanded robust testing infrastructure and statistical validation procedures.

\textbf{Reproducibility Measures}: Maintaining experimental reproducibility while dealing with inherently stochastic agent behavior required careful seed management, environment standardization, and comprehensive logging of all system parameters.

\section{Dataset Processing and Experimental Setup}
\label{sec:dataset-processing}

\subsection{Training Data Preparation}

Preparing \ac{SWE-Gym} (described in Section~\ref{sec:experimental-setup}) for online \ac{RL} training revealed several practical challenges:

\textbf{Environment Consistency}: Ensuring reproducible container environments across the diverse projects in \ac{SWE-Gym} required extensive testing and standardization efforts, as many repositories contained outdated or conflicting dependency specifications.

\textbf{Task Complexity Distribution}: The uneven distribution of task difficulties in \ac{SWE-Gym} necessitated careful sampling strategies to prevent training from concentrating on either trivially simple or impossibly complex examples.

\textbf{Ground Truth Validation}: A significant number of ground truth patches in the original dataset required manual validation and correction, as some did not apply cleanly or introduced unintended side effects when tested in isolated environments.

\subsection{Evaluation Infrastructure}

\textbf{Baseline Establishment}: Creating reliable baseline measurements required extensive infrastructure work to ensure consistent evaluation conditions, including standardization of timeout policies, resource limits, and failure handling procedures.

\textbf{Metric Validation}: Validating the correctness of our evaluation metrics (particularly the patch similarity components described in Section~\ref{sec:reward-design}) required comparison against manual evaluations and cross-validation with existing benchmarks.

\todoinline{Describe specific challenges encountered during evaluation setup, such as environment inconsistencies, timeout handling, and result interpretation difficulties}

\subsection{Experimental Validation Challenges}

\textbf{Result Interpretation}: Determining whether observed improvements resulted from genuine learning versus system optimizations or environmental factors required careful ablation studies and control experiments.

\textbf{Training Stability}: Maintaining stable training dynamics while balancing exploration and exploitation proved challenging, with several experiments requiring restart due to mode collapse or divergent behavior.

\textbf{Statistical Significance}: The high variance inherent in agent-based \ac{RL} training necessitated larger sample sizes and longer training runs than initially anticipated, significantly increasing computational requirements.

\todoinline{Add specific details about hyperparameter sensitivity, training duration challenges, and convergence criteria that were discovered during experimental execution}

\section{Open Source Implementation}
\label{sec:open-source}

A significant outcome of this work is the release of a complete open-source implementation enabling other researchers to reproduce and extend our results.

\subsection{CodeRepairRL Framework}

The complete system has been made available as an integrated framework, including the optimized \ac{GRPO} implementation for coding agents, vLLM integration with live weight synchronization, and evaluation infrastructure.

\textbf{Implementation Challenges}: Packaging the complex distributed system for general use required significant additional engineering effort, including comprehensive documentation, installation automation, and configuration management for diverse computational environments.

\textbf{Community Considerations}: Balancing ease of use with system flexibility demanded careful design decisions about which components to expose as configurable parameters versus hardcoded optimization choices based on our experimental experience.

\subsection{Reproducibility Infrastructure}

Beyond code release, we provide comprehensive training configurations, optimal hyperparameter settings discovered through experimentation, and detailed setup procedures to enable exact reproduction of our results.

\todoinline{Describe specific challenges in making the system accessible to other researchers, documentation requirements, and lessons learned about open-source scientific software development}

\section{Key Implementation Insights and Lessons Learned}
\label{sec:lessons-learned}

The process of implementing and executing online \ac{RL} experiments for coding agents revealed several critical insights that extend beyond the specific technical contributions of this work.

\subsection{System Complexity and Distributed Training}

\textbf{Engineering Effort Underestimation}: The implementation complexity of combining online \ac{RL} with large language models far exceeded initial estimates, with system integration challenges consuming substantially more development time than algorithmic work.

\textbf{Failure Mode Discovery}: Many critical failure modes only emerged during extended training runs, highlighting the importance of robust testing infrastructure and the difficulty of predicting distributed system behavior at scale.

\textbf{Performance Optimization Criticality}: Minor inefficiencies in data handling, memory management, or communication protocols compounded dramatically during long training runs, making systematic optimization essential for practical deployment.

\subsection{Agent Design and Training Dynamics}

\textbf{Tool Interface Sensitivity}: The design of agent tools (detailed in Section~\ref{sec:nano-agent}) proved more influential on training outcomes than initially anticipated, with small changes in interface design significantly affecting learning dynamics and final performance.

\textbf{Reward Signal Quality}: The sparse, terminal reward structure (described in Section~\ref{sec:reward-design}) created substantial training challenges, requiring careful hyperparameter tuning and batch composition strategies to maintain stable learning.

\textbf{Exploration Strategy Impact}: Agent exploration patterns significantly influenced both training efficiency and final capabilities, with naive exploration strategies leading to poor sample efficiency and suboptimal learning outcomes.

\subsection{Practical Deployment Considerations}

\textbf{Compute Resource Management}: The dual requirements of training and inference placed unexpected demands on computational infrastructure, requiring sophisticated resource allocation strategies not present in traditional \ac{ML} workflows.

\textbf{Monitoring and Debugging Complexity}: Debugging distributed agent systems proved substantially more challenging than traditional model training, necessitating extensive logging and visualization infrastructure for effective troubleshooting.

\textbf{Scalability Bottlenecks}: Several system components that functioned adequately at small scales became significant bottlenecks during full-scale experiments, emphasizing the importance of early scalability testing.

\subsection{Research Methodology Implications}

\textbf{Experimental Design Challenges}: The high variance and computational cost of agent-based RL experiments required careful experimental design and statistical analysis procedures not commonly used in standard \ac{ML} research.

\textbf{Reproducibility Requirements}: Ensuring reproducible results across different computational environments required extensive documentation and standardization efforts, highlighting challenges in reproducible research for complex systems.

\textbf{Evaluation Methodology Complexity}: Developing reliable evaluation procedures for interactive agents proved more challenging than anticipated, requiring extensive validation of metrics and careful consideration of evaluation environment consistency.

These insights provide important guidance for future research in online \ac{RL} for coding agents and highlight the substantial engineering investment required to transform theoretical advances into practical, reproducible research contributions.
