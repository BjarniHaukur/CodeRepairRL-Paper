\newpage
\thispagestyle{plain}
~\\
\vfill
{ \setstretch{1.1}
\subsection*{Authors}
Bjarni Haukur Bjarnason <bhbj@kth.se>\\ Electrical Engineering and Computer Science\\ KTH Royal Institute of Technology

\subsection*{Place for Project} Stockholm, Sweden\\ KTH

\subsection*{Examiner} Martin Monperrus <monperrus@kth.se>\\ Division of Theoretical Computer Science\\ KTH Royal Institute of Technology

\subsection*{Supervisor} André Afonso Nunes Silva <andreans@kth.se>\\ Division of Theoretical Computer Science\\ KTH Royal Institute of Technology ~ }

\newpage \thispagestyle{plain} \chapter*{Acronyms}

\begin{acronym}[TerminalBench]
\acro{GPU}{Graphics Processing Unit}
\acro{CPU}{Central Processing Unit}
\acro{CUDA}{Compute Unified Device Architecture}
\acro{NCCL}{NVIDIA Collective Communications Library}

\acro{NAISS}{National Academic Infrastructure for Supercomputing in Sweden}

\acro{AI}{Artificial Intelligence}
\acro{ML}{Machine Learning}
\acro{NLP}{Natural Language Processing}

\acro{API}{Application Programming Interface}
\acro{APR}{Automated Program Repair}

\acro{MDP}{Markov Decision Process}
\acro{RL}{Reinforcement Learning}
\acro{PPO}{Proximal Policy Optimization}
\acro{GRPO}{Group Relative Policy Optimization}
\acro{DAPO}{\textbf{D}ecoupled Clip and \textbf{D}ynamic s\textbf{A}mpling \textbf{P}olicy \textbf{O}ptimization}
\acro{GSPO}{Group Sequence Policy Optimization}
\acro{RLHF}{Reinforcement Learning from Human Feedback}
\acro{KL}{Kullback-Leibler}

\acro{LoRA}{Low-Rank Adaptation}
\acro{SFT}{Supervised Fine-Tuning}

\acro{LLM}{Large Language Model}
\acro{MoE}{Mixture of Experts}

\acro{CLI}{Command Line Interface}
\acro{JSON}{JavaScript Object Notation}
\acro{AST}{Abstract Syntax Tree}
\acro{IDE}{Integrated Development Environment}

\acro{TerminalBench}{A benchmark for terminal-centric agent tasks}
\acro{TauBench}{A benchmark for terminal-oriented agent tasks}
\end{acronym}


\newpage \thispagestyle{plain} \chapter*{Abstract} Software engineering agents have emerged as the dominant approach for automated program repair, with systems like SWE-agent, OpenHands and Claude Code achieving unprecedented success on real-world software engineering tasks.
These agents augment large language models with tools for repository navigation, code execution, and iterative refinement—capabilities essential for solving complex software engineering problems.

% Motivation
Two critical disconnects limit current approaches.
First, a training-deployment mismatch exists: while deployment requires dynamic, multi-step interactions with complex software environments, training typically relies on static datasets of isolated code changes.
Second, a language isolation barrier prevents unified training: test-based rewards require language-specific infrastructure including test runners, build systems, and package managers, making multilingual training prohibitively complex.

Frontier labs have likely addressed the training-deployment gap—systems like Anthropic's Claude Code and OpenAI's Codex demonstrate capabilities suggesting agent-based training—but their methods remain proprietary.
Similarly, while companies like OpenHands have achieved impressive results, they have not released their training procedures, leaving the open-source community without access to these critical techniques.

% Contribution
This thesis presents \textit{one of the first fully open training recipes} for coding agents, addressing a fundamental misalignment in current practice: while deployment combines models with agent scaffolds providing tools, memory, and action spaces, training typically optimizes models in isolation.
We introduce a \ac{RL} approach that trains the \ac{LLM} within its deployment scaffold, enabling the model to learn tool usage and debugging strategies through environmental interaction rather than mere pattern matching.
Our work bridges the gap between closed-source frontier capabilities and open research, providing complete implementation details, training procedures and infrastructure specifications.

% Detail/Nuance
We implement this through \ac{GSPO}~\cite{gspo2025} with an explicit KL regularizer, enabling stable updates while models actively debug real repositories during training.
The key technical innovation lies in our training-inference duality: a custom communication layer enables simultaneous model training and serving, with \ac{NCCL} facilitating live weight synchronization across distributed \acsp{GPU}.
Our \textit{Nano Agent} architecture provides essential bash and file operations, enabling investigation of how agent capabilities develop through \ac{RL} training with minimal tool complexity.

% Evidence / Secondary Contribution
Training on a 1{,}000-task curriculum over 144 \ac{GPU}-hours across 2 days on 3 A100s, tractable within typical academic allocations, demonstrates measurable improvements in harness adaptation and patch-similarity rewards.
Patch completion rates more than double from 37\% to 78\%, while patch-similarity rewards increase by 54\% on SWE-Bench-Verified.
However, test-verified success remains flat at approximately 6--7\%.
Training across nine programming languages demonstrates that execution-free rewards enable language-agnostic learning, with all languages showing measurable improvements throughout training.

% Narrow Impact
For automated program repair, this work demonstrates that online \ac{RL} training substantially improves agent operational reliability within academic compute constraints, though test-verified correctness improvements were not observed at this training scale.
Similar methodologies achieve strong results with substantially more computational resources, suggesting extended training may be necessary.

% Broad Impact
Beyond code repair, \ac{RL} offers a general framework for training \ac{AI} systems on interactive tasks where static datasets poorly capture deployment complexity.
By demonstrating that models can learn rich environmental interactions through \ac{RL} rather than requiring hand-crafted reasoning chains, this work provides one of the first fully open training recipes for coding agents, enabling broader academic investigation of experiential learning approaches.

\subsection*{Keywords}
automated program repair, coding agents, reinforcement learning, group relative policy optimization, group sequence policy optimization, online learning, tool-augmented language models, execution-free training, multilingual generalization

\newpage
\thispagestyle{plain}
%%%%%%%%%%%%%%%%%%%%%%%%%%%%%%%%%%%%
%%	 The Swedish abstract         %%
%%%%%%%%%%%%%%%%%%%%%%%%%%%%%%%%%%%%
\chapter*{Sammanfattning}
%%%%%%%%%%%%%%%%%%%%%%%%%%%%%%%%%%%%

Kodagenter har vuxit fram som den dominerande metoden för automatisk programreparation, där system som SWE-agent, OpenHands och Claude Code uppnår enastående framgång på verkliga mjukvaruutvecklingsuppgifter.
Dessa agenter förstärker stora språkmodeller med verktyg för navigering i kodbaser, kodexekvering och iterativ förfining—förmågor som är nödvändiga för att lösa komplexa mjukvaruutvecklingsproblem.

Två kritiska skillnader begränsar nuvarande metoder.
För det första finns en tränings-implementeringsskillnad: medan implementation kräver dynamiska, flerstegsinteraktioner med komplexa mjukvarumiljöer, förlitar sig träning vanligtvis på statiska dataset av isolerade kodändringar.
För det andra förhindrar en språkisoleringsbarriär enhetlig träning: testbaserade belöningar kräver språkspecifik infrastruktur inklusive testkörare, byggsystem och pakethanterare, vilket gör flerspråkig träning oöverkomligt komplex.

Frontlaboratorier har sannolikt adresserat tränings-implementeringsgapet—system som Anthropics Claude Code och OpenAIs Codex visar förmågor som antyder agentbaserad träning—men deras metoder förblir proprietära.
På liknande sätt har företag som OpenHands uppnått imponerande resultat men har inte släppt sina träningsförfaranden, vilket lämnar öppen källkods-gemenskapen utan tillgång till dessa kritiska tekniker.

Denna avhandling presenterar \textit{ett av de första helt öppna träningsrecepten} för kodagenter, och adresserar en grundläggande skillnad i nuvarande praxis: medan implementation kombinerar modeller med agentskallar som tillhandahåller verktyg, minne och handlingsutrymmen, optimerar träning vanligtvis modeller isolerat.
Vi introducerar en \ac{RL}-metod som tränar \ac{LLM} inom dess implementeringsskall, vilket möjliggör att modellen lär sig verktygsanvändning och felsökningsstrategier genom miljöinteraktion snarare än bara mönsterigenkänning.
Vårt arbete överbryggar klyftan mellan slutna frontierfunktioner och öppen forskning, och tillhandahåller fullständiga implementeringsdetaljer, träningsförfaranden och infrastrukturspecifikationer.

Vi implementerar detta genom \ac{GSPO}~\cite{gspo2025} med en explicit KL-regularisator, vilket möjliggör stabila uppdateringar medan modeller aktivt felsöker verkliga kodbaser under träning.
Den viktigaste tekniska innovationen ligger i vår tränings-inferens-dualitet: ett anpassat kommunikationslager möjliggör samtidig modellträning och servering, med \ac{NCCL} som underlättar live viktsynkronisering över distribuerade \acsp{GPU}.
Vår \textit{Nano Agent}-arkitektur tillhandahåller grundläggande bash- och filoperationer, vilket möjliggör undersökning av hur agentförmågor utvecklas genom \ac{RL}-träning med minimal verktygskomplexitet.

Träning på ett 1{,}000-uppgifters läroplan över 144 \ac{GPU}-timmar under 2 dagar på 3 A100s, genomförbart inom typiska akademiska tilldelningar, visar mätbara förbättringar i skalanpassning och patch-likhetsbelöningar.
Patch-slutförandefrekvenser mer än fördubblas från 37\% till 78\%, medan patch-likhetsbelöningar ökar med 54\% på SWE-Bench-Verified.
Dock förblir testverifierad framgång platt vid cirka 6--7\%.
Träning över nio programmeringsspråk visar att exekveringsfria belöningar möjliggör språkagnostisk inlärning, där alla språk visar mätbara förbättringar under träning.

För automatisk programreparation visar detta arbete att online-\ac{RL}-träning väsentligt förbättrar agentoperativ tillförlitlighet inom akademiska beräkningsbegränsningar, även om testverifierade korrekthetsförbättringar inte observerades vid denna träningsskala.
Liknande metoder uppnår starka resultat med väsentligt mer beräkningsresurser, vilket antyder att utökad träning kan vara nödvändig.

Utöver kodreparation erbjuder \ac{RL} ett allmänt ramverk för träning av \ac{AI}-system på interaktiva uppgifter där statiska dataset dåligt fångar implementeringskomplexitet.
Genom att visa att modeller kan lära sig rika miljöinteraktioner genom \ac{RL} snarare än att kräva handgjorda resonemangskedjor, tillhandahåller detta arbete ett av de första helt öppna träningsrecepten för kodagenter, vilket möjliggör bredare akademisk undersökning av erfarenhetsbaserade inlärningsmetoder.

\subsection*{Nyckelord}
automatisk programreparation, kodagenter, förstärkningsinlärning, grupprelativ policyoptimering, gruppsekvens-policyoptimering, online-inlärning, verktygsförstärkta språkmodeller, exekveringsfri träning, flerspråkig generalisering

\newpage
\thispagestyle{plain}
\chapter*{Acknowledgements}
I would like to express my sincere gratitude to my supervisor, André Afonso Nunes Silva, and my examiner, Martin Monperrus, for their exceptional guidance throughout this thesis.
Their supervision extended beyond technical feedback: they trusted me to pursue an ambitious research direction and supported a complete methodological pivot midway through the project when evidence suggested a more promising path.

This research would not have been possible without access to the computational infrastructure provided by the National Academic Infrastructure for Supercomputing in Sweden (\acs{NAISS}).
Training on the Alvis and Berzelius \ac{GPU} clusters enabled experimentation at a scale far beyond typical academic resources, making the investigation of online \ac{RL} for coding agents feasible.
I am grateful to the infrastructure providers and the Swedish research community for maintaining these shared resources.

I thank the members of ASSERT-KTH who regularly attended the Monday meetings, which operated as an informal book club for those interested in \ac{AI} for software engineering.
These sessions provided a welcoming forum to discuss recent papers, share emerging ideas, and engage with a group of like-minded researchers.

I acknowledge the developers and maintainers of the open-source projects used in this research: TRL and the broader HuggingFace ecosystem, vLLM, DeepSpeed, PyTorch and more.

Lastly, I thank my friends at KTH, Jonas and Navin, who have been with me from the start of my master's studies.
Their steady friendship and camaraderie are the foundation on which these studies stand.

\newpage

\etocdepthtag.toc{mtchapter}
\etocsettagdepth{mtchapter}{subsection}
\etocsettagdepth{mtappendix}{none}
\thispagestyle{plain}
\tableofcontents

\newpage
