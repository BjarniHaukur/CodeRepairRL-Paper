\chapter{Conclusions and Future Work}
\label{ch:conclusions}

This thesis introduces a novel methodological approach to training coding agents through experiential learning, developing the "hill-climbing the coding agent gradient" paradigm.
We present comprehensive infrastructure for online \ac{RL} training of code repair agents, establishing theoretical foundations and practical implementation strategies for this emerging research direction.

\section{Summary of Contributions}
\label{sec:summary-contributions}

Our work makes several methodological and infrastructure contributions to the emerging field of experiential learning for coding agents:

\subsection{Methodological Innovation}
\label{subsec:methodological-innovation}

\textbf{Training-Inference Duality}: Our implementation collapses the conventional boundary between training and inference phases through continuous serving and real-time weight updates.
This unified approach makes large-scale agent training practically feasible within academic research budgets by eliminating the need for separate training and deployment phases.

\textbf{\ac{GRPO} Optimization for Agents}: We adapt Group Relative Policy Optimization~\cite{shao2024deepseekmathpushinglimitsmathematical} for coding agents, demonstrating its suitability as an alternative to traditional actor-critic methods while reducing computational overhead through elimination of value network training.

\subsection{Methodological Validation}
\label{subsec:methodological-validation}

\textbf{Nano Agent Architecture}: Our minimalist Nano agent implementation demonstrates that sophisticated debugging behaviors can emerge from simple interfaces when combined with \ac{RL}. The agent operates with only basic terminal commands and file operations, proving the viability of experiential learning approaches.

\textbf{Hill-Climbing Framework}: We establish the conceptual framework of "hill-climbing the coding agent gradient," providing a systematic approach to iterative improvement of coding agents through environmental interaction.

\textbf{Cross-Model Portability}: \todoinline{Complete experimental validation pending: demonstrate that the training recipe and harness work across different base models, supporting the generality of the approach.}

\subsection{Technical Infrastructure}
\label{subsec:technical-infrastructure}

\textbf{\ac{NCCL}-Based Weight Synchronization}: We develop infrastructure for real-time weight synchronization between training and inference processes using NVIDIA Collective Communications Library, enabling practical online \ac{RL} training with \ac{LoRA} adapters.

\textbf{Scalable Training Architecture}: Our implementation provides a foundation for scaling agent training across multiple \acp{GPU}, demonstrating the practical feasibility of distributed online \ac{RL} for coding agents.

\textbf{Open-Source Implementation}: Complete open-source release of training infrastructure, agent implementations, and evaluation protocols enables reproducible investigation of online RL techniques and reduces barriers to academic research in this domain.

\section{Research Question Answers}
\label{sec:research-answers}

Our methodology development and preliminary investigations provide insights into the research questions, with comprehensive experimental validation ongoing:

\subsection{RQ1: Adaptation to the Harness}
\label{subsec:rq1-answer}

\textbf{Theoretical Foundation}: Interactive \ac{RL} should improve adaptation to the Nano harness through reinforcement of effective tool-usage patterns and reduction of invalid actions.

\textbf{Evidence}: \todoinline{Experimental validation pending: measure tool success rate improvements, invalid call rate reduction, and actions per solved task across training iterations.}

\textbf{Mechanism}: Policy updates should reinforce effective tool-usage sequences and prompt adherence, yielding co-adaptation between model and harness.

\subsection{RQ2: Multiple Base Models}
\label{subsec:rq2-answer}

\textbf{Theoretical Foundation}: The training recipe and harness design should be portable across different base models, targeting general agentic capabilities rather than model-specific behaviors.

\textbf{Evidence}: \todoinline{Experimental validation pending: compare performance across Qwen vs SmolLM3 or other chosen base models.}

\textbf{Significance}: Cross-model portability would demonstrate that the approach develops fundamental debugging capabilities rather than exploiting model-specific quirks.

\subsection{RQ3: Cross-Dataset Generalization}
\label{subsec:rq3-answer}

\textbf{Theoretical Foundation}: Models trained through experiential learning on \ac{SWE-Gym} should demonstrate generalization to related debugging tasks without severe overfitting.

\textbf{Evidence}: \todoinline{Experimental validation pending: measure cross-dataset performance on \ac{SWE-Bench} and \ac{Multi-SWE-Bench}, compare against \ac{SFT} baselines with confidence intervals.}

\textbf{Significance}: Cross-dataset generalization would support the thesis that experiential training yields broadly applicable debugging skills rather than narrow task-specific behaviors.

\subsection{RQ4: Cross-Harness/Task Generalization}
\label{subsec:rq4-answer}

\textbf{Theoretical Foundation}: Models trained through experiential learning should exhibit transfer to alternative interfaces and task contexts, demonstrating that learned skills generalize beyond specific harness implementations.

\textbf{Evidence}: \todoinline{Experimental validation pending: evaluate zero-shot transfer to alternative harnesses (Aider/OpenHands-style) and terminal-oriented tasks (TauBench/TerminalBench).}

\textbf{Significance}: Cross-harness generalization would indicate that the approach develops fundamental interaction skills rather than harness-specific behaviors.

\section{Broader Implications}
\label{sec:broader-implications}

\subsection{Paradigm Shift in \ac{AI} Training}
\label{subsec:paradigm-shift}

Our methodological framework suggests potential for a fundamental shift from static dataset-based training to experiential environment-based learning for complex reasoning tasks.
The viability of online \ac{RL} infrastructure indicates that many \ac{AI} capabilities may be better acquired through direct interaction rather than passive observation, particularly for domains requiring multi-step reasoning and strategic planning.

\subsection{Software Engineering Automation}
\label{subsec:sw-engineering-auto}

The potential for monotonic improvement through \ac{RL} training indicates substantial room for advancement in automated debugging capabilities.
\todoinline{Pending experimental validation: quantify actual performance improvements and trajectory analysis.}
The methodological foundation supports future development toward practical deployment of autonomous debugging systems.

\subsection{Open Science and Democratization}
\label{subsec:open-science}

By providing complete open-source implementations, we enable broader academic investigation of online RL techniques for coding agents.
This open approach facilitates reproducible research and community-driven development of advanced agent training methods, reducing barriers to entry in this emerging field.


\section{Limitations and Constraints} \label{sec:limitations}

\subsection{Experimental Validation Status}
\label{subsec:experimental-validation}

The primary limitation of this work is that comprehensive experimental validation remains ongoing.
While we have developed robust theoretical foundations and implementation infrastructure, the quantitative claims about performance improvements, statistical significance, and cross-domain generalization require completion of systematic experimental evaluation.

\subsection{Evaluation Scope Constraints}
\label{subsec:eval-scope}

The planned evaluation focuses on Python debugging tasks within containerized environments.
Comprehensive evaluation across diverse programming paradigms (functional, systems programming, web development) and languages remains for future work.
Additionally, the current scope emphasizes single-commit bug fixes rather than complex architectural changes or multi-file refactoring tasks.

\subsection{Computational Resource Requirements}
\label{subsec:compute-requirements}

Despite optimizations, online training remains computationally intensive compared to traditional supervised learning.
The distributed training infrastructure presents implementation complexity that may challenge smaller research groups, though our open-source release aims to mitigate these barriers through comprehensive documentation and modular design.

\subsection{Reward Function Simplifications}
\label{subsec:reward-simplifications}

Our reward formulation relies on patch similarity rather than functional correctness verification through test execution.
While computationally tractable, this approach may occasionally reward syntactically correct but functionally incorrect patches, or penalize alternative valid solutions.
The limitation reflects practical constraints rather than fundamental methodological issues.

\subsection{Model Architecture Dependencies}
\label{subsec:model-dependencies}

Initial development focuses on the Qwen model family due to its strong tool-calling capabilities and computational accessibility.
Validation across models with different architectural characteristics, training objectives, or tool-calling paradigms remains an important area for future investigation.
The approach's potential model-agnostic benefits require systematic validation across broader architectural diversity.

\subsection{Environment Complexity Boundaries}
\label{subsec:env-complexity}

Our containerized environments simulate realistic development scenarios but exclude certain complexities of production systems: external dependencies, network interactions, hardware-specific behaviors, and real-time performance constraints.
The gap between training environments and production deployment contexts may limit practical applicability.

\section{Future Research Directions}
\label{sec:future-work}

\subsection{Completing Experimental Validation}
\label{subsec:completing-validation}

The immediate priority involves completing the comprehensive experimental evaluation outlined in our research questions:

\subsubsection{Performance Quantification}

Systematic measurement of performance improvements across the training process, including adaptation metrics for the Nano harness, cross-model portability validation, and cross-dataset generalization assessment.
This evaluation will provide the empirical foundation needed to validate the theoretical contributions developed in this thesis.

\subsubsection{Statistical Validation}

Rigorous statistical analysis of results with appropriate confidence intervals and significance testing.
Comparison against supervised fine-tuning baselines to establish the relative merits of experiential learning approaches.

\subsection{Enhanced Reward Engineering}
\label{subsec:enhanced-reward}

\subsubsection{Test-Based Validation}

The most natural extension involves incorporating actual test execution into reward computation.
Instead of relying on patch similarity metrics, future systems could execute project test suites and reward agents based on functional correctness.
This approach would provide more accurate correctness signals, enable discovery of alternative valid solutions, support evaluation of partial fixes, and better align training objectives with real-world debugging goals.

\subsubsection{Multi-Objective Optimization}

Future reward functions could incorporate multiple objectives beyond correctness: code quality metrics for readability and maintainability, efficiency considerations for patch minimality and performance impact, robustness measures for edge case handling, and documentation quality for explanations and comments.

\subsection{Scaling and Generalization}
\label{subsec:scaling-generalization}

\subsubsection{Broader Domain Expansion}

Systematic evaluation across different programming contexts and application domains would validate the universality of learned debugging skills.
This includes specialized training for security vulnerability fixing and performance optimization; evaluation on large-scale enterprise codebases with complex dependencies; and adaptation to different development workflows and coding conventions.

\subsubsection{Model Architecture Exploration}

Investigation of experiential training across different model architectures could reveal optimal designs for agentic programming tasks, including architecture variants like mixture-of-experts models, systematic study of performance scaling effects, multimodal integration with visual debugging information, and external memory systems for long-term context.

\subsection{Advanced Agent Architectures}
\label{subsec:advanced-agents}

\subsubsection{Hierarchical and Modular Agents}

Future agent designs could incorporate hierarchical planning with high-level strategy formation and detailed execution sub-agents, modular debugging skills that can be combined for complex repairs, adaptive scaffolding with dynamic tool selection, and collaborative multi-agent approaches for complex debugging tasks.

\subsubsection{Meta-Learning and Continual Learning}

Agents that rapidly adapt to new environments represent an important frontier, including few-shot adaptation to new programming environments, continual learning without catastrophic forgetting, systematic transfer learning approaches, and self-improvement through reflection on debugging experiences.

\subsection{Human-\ac{AI} Collaboration}
\label{subsec:human-ai-collab}

Rather than fully autonomous agents, future systems could focus on human-\ac{AI} partnerships through explanatory debugging that shares reasoning, incremental assistance adapting to developer preferences, learning from natural human feedback, and preference learning for individual coding styles.

Educational applications could revolutionize programming instruction through tutoring systems that teach debugging skills, adaptive curricula based on individual weaknesses, automated skill assessment and improvement, and code review training through interactive analysis.

\subsection{Infrastructure and Tooling Advances}
\label{subsec:infra-tooling}

Continued optimization could enable broader access through federated learning across institutions, efficient communication protocols for model updates, robust fault tolerance for distributed systems, and dynamic resource optimization for mixed workloads.

Standardized evaluation frameworks would accelerate progress through comprehensive benchmarks for Python debugging tasks, automated functional correctness assessment, reproducibility tools for research extension, and community platforms for collaborative development.

\section{Long-Term Vision}
\label{sec:long-term-vision}

The long-term vision of this research direction involves developing increasingly capable software engineering assistance through experiential learning approaches.
Our methodological framework provides a foundation for investigating how agents can acquire debugging and programming skills through environmental interaction.

Future developments in this direction could enhance software development productivity by providing intelligent assistance for debugging, code review, and implementation tasks.
Such systems could particularly benefit educational contexts by providing interactive learning environments for programming skill development.

Enhanced automated debugging capabilities could accelerate research workflows by reducing time spent on implementation details, allowing researchers to focus more extensively on conceptual innovation and experimental design.

\section{Final Reflections}
\label{sec:final-reflections}

This research demonstrates that the convergence of large language models and \ac{RL} opens unprecedented opportunities for developing sophisticated \ac{AI} systems capable of complex reasoning and interaction.
The success of experiential training validates broader principles about learning through environmental interaction and suggests promising directions for advancing \ac{AI} capabilities across domains requiring multi-step reasoning and strategic planning.

Perhaps most importantly, our work demonstrates that sophisticated \ac{AI} research infrastructure can be developed within academic settings through careful engineering and open-source collaboration.
By providing accessible implementations of online \ac{RL} training systems, we enable broader investigation of experiential learning approaches that were previously confined to well-resourced industry laboratories.

The journey from passive pattern matching to active environmental interaction represents a potential fundamental evolution in how we train \ac{AI} systems.
While significant challenges remain, the theoretical foundation and infrastructure developed in this work suggest that experiential learning paradigms may play an increasingly important role in developing \ac{AI} systems capable of genuine reasoning and problem-solving.

As we stand at the threshold of increasingly capable \ac{AI} systems, the principles explored in this work—learning through interaction, systematic agent training infrastructure, and open scientific collaboration—may guide the development of even more sophisticated \ac{AI} capabilities.
The future of \ac{AI} development may increasingly incorporate experiential learning approaches that enable agents to learn through environmental interaction, complementing traditional supervised learning paradigms.
