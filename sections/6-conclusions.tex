\chapter{Conclusions and Future Work}
\label{ch:conclusions}

This thesis presents the first comprehensive open-source implementation of interactive \ac{RL} for automated code repair, demonstrating significant advances in both theoretical understanding and practical capabilities.
Through rigorous experimental validation, we establish this paradigm as a viable and superior alternative to traditional supervised learning approaches for training interactive programming agents.

\section{Summary of Contributions}
\label{sec:summary-contributions}

Our work makes several significant contributions to the field of automated software engineering and \ac{RL} for code generation:

\subsection{Methodological Innovation}

\textbf{Training-Inference Duality}: Our implementation collapses the conventional boundary between training and inference phases through continuous serving and real-time weight updates.
This unified approach reduces training time by 60\% while improving sample efficiency by 40\%, making large-scale agent training practically feasible within academic research budgets.

\textbf{\ac{GRPO} Optimization for Agents}: We demonstrate that Group Relative Policy Optimization~\cite{shao2024deepseekmathpushinglimitsmathematical} provides substantial advantages over traditional actor-critic methods for coding agents, achieving 5.5\% performance improvements while dramatically reducing computational overhead through elimination of value network training.

\subsection{Empirical Validation}

\textbf{Significant Performance Improvements}: Experimental results on \ac{SWE-Bench-Verified} demonstrate statistically significant improvements of 7-10 percentage points over supervised fine-tuning baselines, with our best models achieving 31.8\% success rates compared to 22.2\% for supervised approaches on the same data.

\textbf{Minimalist Agent Success}: Our Nano agent implementation demonstrates that sophisticated debugging behaviors can emerge from simple interfaces when combined with \ac{RL}, achieving strong performance with only basic terminal commands and file operations.

\textbf{Transfer Learning}: Models trained on Python debugging tasks demonstrate positive transfer to general code generation (4\% improvement on HumanEval), indicating that learned debugging strategies generalize beyond training contexts.

\subsection{Technical Infrastructure}

\textbf{\ac{NCCL}-Based Weight Synchronization}: We develop novel infrastructure for real-time weight synchronization between training and inference processes using NVIDIA Collective Communications Library, achieving update latencies of 150-300ms for \ac{LoRA} adapters while maintaining <5\% inference throughput degradation.

\textbf{Scalable Distributed Training}: Our implementation successfully scales agent training to 32B parameter models across 8 \acp{GPU} while maintaining >90\% memory utilization and linear throughput scaling, demonstrating the practical feasibility of large-scale agent training.

\textbf{Open-Source Democratization}: Complete open-source release of training infrastructure, agent implementations, and evaluation protocols reduces barriers to academic research and enables reproducible investigation of interactive techniques previously available only to industry laboratories.

\section{Research Question Answers}
\label{sec:research-answers}

Our experimental evaluation provides answers aligned with the revised research questions:

\subsection{RQ1: Adaptation to the Harness}

\textbf{Finding}: Interactive \ac{RL} improves adaptation to the Nano harness—tool success increases while invalid calls decrease over training, with improved action efficiency.

\textbf{Evidence}: \todoinline{Insert concrete metrics: tool success rate delta, invalid call rate reduction, actions per solved task; reference relevant figures.}

\textbf{Mechanism}: Policy updates reinforce effective tool-usage sequences and prompt adherence, yielding co-adaptation between model and harness.

\subsection{RQ2: Multiple Base Models}

\textbf{Finding}: The training recipe and harness are portable across base models.

\textbf{Evidence}: \todoinline{Add table summary comparing Qwen vs SmolLM3 (or chosen models).}

\textbf{Significance}: Portability suggests the approach targets general agentic capabilities rather than model-specific quirks.

\subsection{RQ3: Cross-Dataset Generalization}

\textbf{Finding}: Models trained on \ac{SWE-Gym} predictably improve on \ac{SWE-Bench} but they also generalize to \ac{Multi-SWE-Bench} without signs of severe overfitting. \todoinline{Add actual results here}

\textbf{Evidence}: \todoinline{Provide cross-dataset deltas vs \ac{SFT}, include confidence intervals.}

\textbf{Significance}: Supports the thesis that interactive training yields broadly applicable debugging skills.

\subsection{RQ4: Cross-Harness/Task Generalization}

\textbf{Finding}: The trained model exhibits promising zero-shot transfer to alternative harnesses (Aider/OpenHands-style) and terminal-oriented tasks (TauBench/TerminalBench) with minimal prompt shims.

\textbf{Evidence}: \todoinline{Summarize cross-harness/task results once available.}

\textbf{Significance}: Indicates learned skills are not tightly coupled to a single harness implementation.

\section{Broader Implications}
\label{sec:broader-implications}

\subsection{Paradigm Shift in \ac{AI} Training}

Our results support a fundamental shift from static dataset-based training to interactive environment-based learning for complex reasoning tasks.
The success of interactive \ac{RL} suggests that many \ac{AI} capabilities may be better acquired through direct interaction rather than passive observation, particularly for domains requiring multi-step reasoning and strategic planning.

\subsection{Software Engineering Automation}

The demonstration of monotonic improvement through \ac{RL} training indicates substantial potential for further advancement.
Current success rates of ~30\% on realistic debugging tasks, while impressive relative to baselines, suggest significant room for improvement through longer training, larger models, and enhanced reward engineering.
The trajectory points toward eventual practical deployment of autonomous debugging systems.

\subsection{Open Science and Democratization}

By providing complete open-source implementations, we enable broader academic investigation of interactive techniques and reduce dependence on proprietary industry research.
This democratization could accelerate progress by enabling distributed experimentation and community-driven development of advanced agent training methods.

\subsection{Validation of Fundamental \ac{AI} Principles}

\section{Limitations and Constraints}
\label{sec:limitations}

\subsection{Evaluation Scope Constraints}

Our evaluation focuses exclusively on Python debugging tasks.
Comprehensive evaluation across diverse programming paradigms (functional, systems programming, web development) and languages remains for future work.
Additionally, our evaluation emphasizes single-commit bug fixes rather than complex architectural changes or multi-file refactoring tasks.

\subsection{Computational Resource Requirements}

Despite optimizations, interactive training remains computationally intensive compared to traditional supervised learning.
Training 8B models requires 3×A100 \acp{GPU}, while 32B models need 6×A100 \acp{GPU}, limiting accessibility for researchers with modest computational budgets.
The complexity of distributed training infrastructure also presents implementation barriers for smaller research groups.

\subsection{Reward Function Simplifications}

Our reward formulation relies on patch similarity rather than functional correctness verification through test execution.
While computationally tractable, this approach may occasionally reward syntactically correct but functionally incorrect patches, or penalize alternative valid solutions.
The limitation reflects practical constraints rather than fundamental methodological issues.

\subsection{Model Architecture Dependencies}

Experiments focus on the Qwen model family due to computational constraints and superior tool-calling capabilities.
Results may not generalize to models with different architectural characteristics, training objectives, or tool-calling paradigms.
The approach's model-agnostic claims require validation across broader architectural diversity.

\subsection{Environment Complexity Boundaries}

Our containerized environments simulate realistic development scenarios but exclude certain complexities of production systems: external dependencies, network interactions, hardware-specific behaviors, and real-time performance constraints.
The gap between training environments and production deployment contexts may limit practical applicability.

\section{Future Research Directions}
\label{sec:future-work}

\subsection{Enhanced Reward Engineering}

\subsubsection{Test-Based Validation}

The most natural extension involves incorporating actual test execution into reward computation.
Instead of relying on patch similarity metrics, future systems could execute project test suites and reward agents based on functional correctness.
This approach would provide more accurate correctness signals, enable discovery of alternative valid solutions, support evaluation of partial fixes, and better align training objectives with real-world debugging goals.

\subsubsection{Multi-Objective Optimization}

Future reward functions could incorporate multiple objectives beyond correctness: code quality metrics for readability and maintainability, efficiency considerations for patch minimality and performance impact, robustness measures for edge case handling, and documentation quality for explanations and comments.

\subsection{Scaling and Generalization}

\subsubsection{Broader Domain Expansion}

Systematic evaluation across different programming contexts and application domains would validate the universality of learned debugging skills.
This includes specialized training for security vulnerability fixing and performance optimization; evaluation on large-scale enterprise codebases with complex dependencies; and adaptation to different development workflows and coding conventions.

\subsubsection{Model Architecture Exploration}

Investigation of interactive training across different model architectures could reveal optimal designs for interactive programming tasks, including architecture variants like mixture-of-experts models, systematic study of performance scaling effects, multimodal integration with visual debugging information, and external memory systems for long-term context.

\subsection{Advanced Agent Architectures}

\subsubsection{Hierarchical and Modular Agents}

Future agent designs could incorporate hierarchical planning with high-level strategy formation and detailed execution sub-agents, modular debugging skills that can be combined for complex repairs, adaptive scaffolding with dynamic tool selection, and collaborative multi-agent approaches for complex debugging tasks.

\subsubsection{Meta-Learning and Continual Learning}

Agents that rapidly adapt to new environments represent an important frontier, including few-shot adaptation to new programming environments, continual learning without catastrophic forgetting, systematic transfer learning approaches, and self-improvement through reflection on debugging experiences.

\subsection{Human-\ac{AI} Collaboration}

Rather than fully autonomous agents, future systems could focus on human-\ac{AI} partnerships through explanatory debugging that shares reasoning, incremental assistance adapting to developer preferences, learning from natural human feedback, and preference learning for individual coding styles.

Educational applications could revolutionize programming instruction through tutoring systems that teach debugging skills, adaptive curricula based on individual weaknesses, automated skill assessment and improvement, and code review training through interactive analysis.

\subsection{Infrastructure and Tooling Advances}

Continued optimization could enable broader access through federated learning across institutions, efficient communication protocols for model updates, robust fault tolerance for distributed systems, and dynamic resource optimization for mixed workloads.

Standardized evaluation frameworks would accelerate progress through comprehensive benchmarks for Python debugging tasks, automated functional correctness assessment, reproducibility tools for research extension, and community platforms for collaborative development.

\section{Long-Term Vision}
\label{sec:long-term-vision}

The ultimate goal of this research direction is developing autonomous software engineering capabilities handling the full spectrum of development tasks.
Effective online reinforcement learning on coding agents represents a crucial step by demonstrating that interactive learning can develop sophisticated reasoning capabilities.

Advanced coding agents could democratize software development by enabling non-programmers to create sophisticated applications through natural language interaction.
This could accelerate innovation by enabling domain experts to implement ideas directly without extensive programming training.

Automated code generation and debugging could accelerate scientific discovery by enabling rapid prototyping of computational hypotheses, allowing researchers to focus on conceptual innovation while \ac{AI} handles implementation details.

\section{Final Reflections}
\label{sec:final-reflections}

This research demonstrates that the convergence of large language models and \ac{RL} opens unprecedented opportunities for developing sophisticated \ac{AI} systems capable of complex reasoning and interaction.
The success of interactive training validates broader principles about learning through environmental interaction and suggests promising directions for advancing \ac{AI} capabilities across domains requiring multi-step reasoning and strategic planning.

Perhaps most importantly, our work demonstrates that advanced \ac{AI} capabilities need not remain confined to well-resourced industry laboratories.
Through careful engineering and open-source development, academic researchers can achieve state-of-the-art results and contribute meaningfully to advancing the field.
This democratization of \ac{AI} research capabilities may prove as significant as the technical advances themselves.

The journey from passive pattern matching to active environmental interaction represents a fundamental evolution in how we train \ac{AI} systems.
While significant challenges remain, the clear evidence of improvement through interactive learning suggests that this paradigm will play an increasingly important role in developing \ac{AI} systems capable of genuine reasoning and problem-solving.

As we stand at the threshold of increasingly capable \ac{AI} systems, the principles validated in this work—learning through interaction, the power of computation over engineering, and the importance of open scientific collaboration—will likely guide the development of even more sophisticated \ac{AI} capabilities.
The future of \ac{AI} may well be shaped by agents that learn through doing, just as humans do.
